\chapter{Comunicazione}
\noindent La comunicazione tra le varie entità del sistema è eseguita tramite \textbf{primitive bloccanti} per \texttt{read} e per \texttt{write}, nel caso di Java, \texttt{writeLine} e \texttt{readLine}. Lo scambio di messaggi avviene in \textbf{locale} basandosi sul protocollo \textbf{TCP/IP} tramite l'utilizzo di socket \textbf{bidirezionali}; pertanto, sarà sufficiente utilizzare una singola socket per ogni client connesso.\\
\\
\noindent Quando viene \textbf{inviato} un messaggio, questo viene posto in un \textbf{buffer interno}: qualora il buffer dovesse essere \textit{pieno}, il thread è bloccato finché non si \textit{svuota}.\\
Al fine di garantire questo comportamente, viene utilizzato il tipo \texttt{PrintWriter}, configurato con \texttt{autoFlush}.\\
\textit{N.B.:} il messaggio viene processato prima di essere inviato, aggiungendo simboli di escape.\\
\\
Quando viene \textbf{ricevuto} un messaggio, il thread è bloccato in attesa di un messaggio da leggere, una volta che il messaggio è presente sullo stream di input, viene processato rimuovendo i simboli di escape e inserito in una struttura dati \texttt{JSON}.\\
\\
\noindent Per design, è assunto che il contenuto e la struttra dei messaggi in JSON siano validi e NON è per tanto necessario eseguire operazioni di controllo.\\
Verranno però validati i parametri inviati dal client.

\section{$\text{Main Client} \Leftrightarrow \text{Handler Server}$}
\noindent Il client principale si interfaccia con l’Handler Server per autenticarsi, e per la comunicazione con gli altri server.\\
L’istante in cui un client si connette all’Handler Server, questo genera automaticamente altre due connessioni ai server ausiliari; a questo punto i tre server si mettono in attesa di messaggi.\\
La comunicazione diretta per l’autenticazione viene effettuata specificando nel campo \texttt{type} della richiesta “generic”.
\section{Autenticazione - \texttt{handleSignInRequest}}
\noindent All’avvio del client, viene mostrata una finestra di login dove l’utente potrà inserire il suo username e la sua password, una volta confermati i dati, l’applicazione costruirà una richiesta \texttt{JSON} con i parametri necessari al login e verrà inviata immediatamente al server.\\Il client si mette poi in attesa di una risposta dal server, se questa contiene i privilegi di accesso allora il login avrà avuto successo, altrimenti la risposta conterrà un errore con una breve spiegazione del problema.

\section{Registrazione - \texttt{handleSignUpRequest}}
\noindent Dalla schermata di login, viene mostrato un bottone che permette di registrarsi, la registrazione richiede all’utente di inserire:
\begin{itemize}
    \item Username;
    \item Password;
    \item Nome;
    \item Cognome.
\end{itemize}
Una volta confermati i dati, l’applicazione costruisce una richiesta \texttt{JSON} con i parametri necessari per la registrazione e la invia immediatamente al server.\\
A questo punto, il client si mette in attesa di una risposta dal server. Se la risposta contiene un \texttt{JSON} vuoto, l’utente viene registrato nel sistema. In caso contrario, la risposta conterrà un errore con una breve spiegazione del problema (es. l’username scelto è già in uso).

\section{$\text{Main Client} \Leftrightarrow \text{Handler Server}$}
\noindent L’Handler Server gestisce la comunicazione con il Poster Server internamente e in maniera trasparente all’utente. I messaggi aventi il campo type impostato a “poster” vengono inoltrati automaticamente a questo server.\\
Le richieste supportate dal Poster Server sono:
\begin{itemize}
    \item \texttt{getHouses};
    \item \texttt{insertHouse};
    \item \texttt{deleteHouse}.
\end{itemize}

\section{Richiesta di Immobili – \texttt{getHouses}}
\noindent Questa richiesta viene effettuata immediatamente quando \texttt{MainViewController}, ossia la pagina principale dell’applicazione, viene mostrata; oppure quando si cambia pagina verso \texttt{MainViewController}.\\
Questa richiesta non ha bisogno di parametri e si limita a restituire l’intera collezione di immobili disponibili nel database come un array \texttt{JSON}.\\
Non essendo previsti errori per questa richiesta, possiamo assumere che andrà sempre a buon fine.

\section{Inserimento di Immobili – \texttt{insertHouse}}
\noindent La richiesta di inserimento di un immobile gestisce sia l’inserimento che la modifica degli stessi. Viene originata da \texttt{AddHouseController}, ossia la pagina di inserimento e modifica immobili.\\
\\
La richiesta dovrà contenere tutti i parametri necessari per l’inserimento di un immobile, con eventuali parametri opzionali come le immagini da inserire e l’id per l’immobile.\\
Quest'ultimo risulta opzionale in quanto, quando la richiesta contiene l’id, viene interpretata come una modifica.\\
\textit{N.B.:} Qualora un immobile non abbia una o più immagini, ne verrà utilizzata una placeholder.\\
Le immagini sono trasmesse dal server e al client in formato \textbf{base64}.\\
Non essendo previsti errori per questa richiesta, possiamo assumere che andrà sempre a buon fine.

\section{Cancellazione di Immobili – \texttt{deleteHouse}}
\noindent La richiesta di cancellazione di un immobile è originata da \texttt{AddHouseController} e necessita solo di un parametro, l’id dell’immobile da eliminare.\\
Poiché l’amministratore non specifica mai manualmente l’id (in quanto automaticamente determinato da \texttt{AddHouseController}), si può assumere che questo sia sempre valido, pertanto, tale funzione non restituisce nessun errore.

\section{$\text{Handler Server} \Leftrightarrow \text{Appointment Server}$}
\noindent L’Handler Server gestisce la comunicazione con l’Appointment Server internamente e in maniera \textit{trasparente} all’utente. I messaggi che hanno il campo \texttt{type} impostato ad “appointment” vengono inoltrati automaticamente a questo server.\\

\section{Richiesta Appuntamenti – \texttt{getAppointmentsForUser / Agent}}
\noindent Questa richiesta acquisisce gli appuntamenti per l’utente specificato, il quale può essere sia un cliente che un agente immobiliare. La richiesta è originata da \texttt{MainViewController} automaticamente.\\
In base al tipo di utente attualmente loggato, viene richiamata la funzione\\ \texttt{getAppointmentsForUser} o \texttt{getAppointmentsForAgent}.\\
Sarà ritornata una lista di appuntamenti presi dall’utente oppure la lista di appuntamenti a cui partecipa l’agente sottoforma di array \texttt{JSON}. Si può inoltre assumere che non saranno ritornati errori.

\section{Richiesta Appuntamenti – insertAgentAvailability}
\noindent Questa richiesta è originata da \texttt{InsertAvailabilityController} ed inserisce un range di disponibilità in base ai parametri \texttt{start\_date} e \texttt{end\_date}.\\
Le date saranno inserite una ad una nel database e, nel caso una data sia già presente, sarà semplicemente saltata.\\
Poiché il controllo \texttt{end\_date > start\_date} è effettuato dal client, si può assumere che non saranno ritornati errori.

\section{Richiesta Appuntamenti – \texttt{bookAppointment}}
\noindent Questa richiesta è originata da \texttt{HouseDetailsController}, in questo caso, l’utente loggato, ossia il cliente, vuole riservare un appuntamento per un immobile scegliendo il primo agente disponibile.\\
La funzione ritorna due errori che devono essere obbligatoriamente gestiti dal client; in entrambi i casi l’appuntamento NON è prenotato e la disponibilità dell’agente rimane invariata.
Gli errori sono:
\begin{itemize}
    \item \textbf{\texttt{notAvailable}:} Se l’utente sceglie una data dove nessun agente ha dato disponibilità;
    \item \textbf{\texttt{alreadyBooked}:} Se l’utente sceglie una data per cui ha già preso un appuntamento. 
\end{itemize}

\section{Richiesta Appuntamenti – removeAppointment}
\noindent Questa richiesta è originata da \texttt{MainViewController} quando l’utente, ossia il cliente, vuole cancellare un appuntamento. La funzione richiede un solo parametro: l’id dell’appuntamento da rimuovere.\\
Poiché l’utente può solo cancellare un appuntamento che appare nella lista di appuntamenti, si può assumere che l’id sarà sempre valido, pertanto la richiesta non ritorna errori.