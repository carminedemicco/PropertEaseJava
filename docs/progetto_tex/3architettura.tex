\chapter{Architettura}
\noindent Il sistema impiega un’architettura di tipo \textbf{ibrido}, utilizzando sia caratteristiche del modello \textbf{client/server} chee quelle del modello \textbf{peer-to-peer}.\\
In dettaglio, la struttura prevede l’utilizzo di tre server:
\begin{itemize}
    \item \textbf{Handler Server:} coordina la comunicazione tra il client e gli altri due server;
    \item \textbf{Poster Server:} svolge funzioni di gestione degli \textit{annunci};
    \item \textbf{Appointment Server:} svolge funzioni di gestione degli \textit{appuntamenti}.
\end{itemize}
La connessione è effettuata tramite \textbf{Socket}, con protocollo \textbf{TCP/IP}, risultando quindi \textit{connection-oriented} e \textit{reliable}.
Possiamo dividere l’architettura in quattro processi:
\begin{itemize}
    \item \textbf{Main Client:} agisce puramente da client e si interfaccia esclusivamente con il processo “Handler Server”, sia per l’autenticazione che per la comunicazione con gli altri due processi;
    \item \textbf{Handler Server:} agisce sia da server, verso il processo “Handler Server” che da client, verso di processi “Poster Server” e “Appointment Server”.\\
    Si occupa principalmente di smistare i messaggi in arrivo dal processo “Main Client” e di gestire l’autenticazione con lo stesso;
    \item \textbf{Poster Server:} agisce puramente da server, verso il processo “Handler Server”.\\
    Si occupa di gestire i comandi riguardanti gli immobili, ossia inserimento, cancellazione e selezione.;
    \item \textbf{Appointment Server:} agisce puramente da server, verso il processo “Handler Server”.\\
    Si occupa di gestire i comandi riguardanti gli appuntamenti, ossia inserimento, cancellazione e selezione..
\end{itemize}
